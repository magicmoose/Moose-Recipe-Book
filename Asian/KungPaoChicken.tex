\recipe{Kung Pao Chicken}
\ingred{
\begin{tabular}{l l}
&marinade\\
&\\
1 tbls & soy sauce \\
2 tsp &  ShaoXing wine \\
1 \sfrac{1}{2} tsp & cornstarch\\
1 lb & chicken breast cubes 1 in\\
&\\
& sauce \\
&\\
1 tbls & black vinegar \\
1 tsp & soy sauce \\
1 tsp & hoisin sauce \\
1 tsp & sesame oil \\
2 tsp & sugar \\
1 tsp & cornstarch\\
 \sfrac{1}{2} tsp& ground Sichuan pepper\\
2 tbls & peanut or vegetable oil\\
8 to 10& dried red chilies\\
3 & scallions, white and green parts separated, thinly sliced\\
2 & garlic cloves, minced\\
1 tsp & minced or grated fresh ginger\\
 \sfrac{1}{4} cup &unsalted dry-roasted peanuts\\
\end{tabular}
}

Marinate the chicken: In a medium bowl, stir together the soy sauce, rice wine, and cornstarch until the cornstarch is dissolved. Add the chicken and stir gently to coat. Let stand at room temperature for 10 minutes.

Prepare the sauce: In another bowl, combine the black vinegar, soy sauce, hoisin sauce, sesame oil, sugar, cornstarch, and Sichuan pepper. Stir until the sugar and cornstarch are dissolved and set aside.

You may need to turn on your stove's exhaust fan, because stir-frying dried chilies on high heat can get a little smoky. Heat a wok or large skillet over high heat until a bead of water sizzles and evaporates on contact. Add the peanut oil and swirl to coat the base. Add the chilies and stir-fry for about 30 seconds, or until the chilies have just begun to blacken and the oil is slightly fragrant. Add the chicken and stir-fry until no longer pink, 2 to 3 minutes.

Add the scallion whites, garlic, and ginger and stir-fry for about 30 seconds. Pour in the sauce and mix to coat the other ingredients. Stir in the peanuts and cook for another 1 to 2 minutes. Transfer to a serving plate, sprinkle the scallion greens on top, and serve.
